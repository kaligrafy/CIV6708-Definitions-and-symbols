\documentclass{article}
\usepackage[colorlinks = true,
            linkcolor = blue,
            urlcolor  = blue,
            citecolor = blue,
            anchorcolor = blue]{hyperref}

\usepackage{longtable}
\usepackage{tabulary}
\usepackage{multirow}
\usepackage{amsmath}
\usepackage[none]{hyphenat}
\usepackage[utf8]{inputenc}
\usepackage{array}
\usepackage{mathtools}
\usepackage{amsmath}
\usepackage{tabularx}
\usepackage{makecell}
\usepackage{exercise}
\usepackage[letterpaper, portrait, top=1in, left=1in, right=1in, bottom=1in, includefoot]{geometry}

\begin{document}

\newcolumntype{C}[1]{>{\centering\arraybackslash}m{#1}}   %% centered
\newcolumntype{R}[1]{>{\raggedleft\arraybackslash}m{#1}}  %% right aligned
\newcolumntype{L}[1]{>{\raggedright\arraybackslash}m{#1}}  %% left aligned
\setlength\LTleft{0pt}
\setlength\LTright{0pt}
\setlength\LTcapwidth{\textwidth}
\setlength{\abovedisplayskip}{0.1pt}
\setlength{\belowdisplayskip}{0.1pt}
\renewcommand{\arraystretch}{1.2}
\newdimen\NetTableWidth

\title{Transport collectif • Définitions et symboles}

\author{Pierre-Léo Bourbonnais et étudiants du cours CIV6708}

\section*{Exercices}

\begin{Exercise}[name=Exercice, origin={Vukan R. Vuchic}]
    Une ligne de bus a une distance inter-arrêt fixe de 250m. La vitesse programmée (vitesse de croisière du bus entre les arrêts) est de 36 km/h. Le temps d'arrêt est de 20 secondes. Les passagers sont uniformément distribués sur le parcours de la ligne et marchent à 5 km/h. On décide de retirer des arrêts pour obtenir une distance inter-arrêts de 500 m. Quel sera le changement de temps de parcours total moyen par passager si la distance moyenne parcourue par passager sur la ligne est de a) 4 km, b) 6 km, c) 8 km, d) 10 km ?
    
    \textit{A bus line has stops 250 m apart. Programmed speed (cruising speed between stops) is 36 km/h. Stop time is 20 seconds. Passengers are uniformly distributed along the line and their walking speed is 5 km/h. We decide to reduce the number of stops to get a distance of 500 m between stops. How would a passenger total travel time change if the average distance traveled on line is a) 4 km, b) 6 km, c) 8 km, d) 10 km ?}
\end{Exercise}

%%\begin{Exercise}[name=Exercice, origin={Vukan R. Vuchic}]
%%    Une ligne de bus a une lngueur de 12 km. Cette ligne possède un arrêt à chaque intersection de blocs et il y a 5 blocs par km. La vitesse du moyenne du trafic est de 34 km/h. Le temps de battement est de 9 minutes à chaque extrémité. Prendre pour acquis que les bus arrêtent à chaque arrêt pendant 20 secondes (incluant 15 sec. d'arrêt et 5 sec. d'accélération/freinage) ? On veut réduire la densité d'arrêts à un arrêt par deux intersections. Le temps d'arrêt augmentera à 25 sec. incluant accélération et décélération de 5 sec. 
    
%%    \textit{A bus line has stops 250 m apart. Programmed speed (cruising speed between stops) is 36 km/h. Stop time is 20 seconds. Passengers are uniformly distributed along the line and their walking speed is 5 km/h. We decide to reduce the number of stops to get a distance of 500 m between stops. How would a passenger total travel time change if the average distance traveled on line is a) 4 km, b) 6 km, c) 8 km, d) 10 km ?}
%%\end{Exercise}

\noindent
  \NetTableWidth=\dimexpr
    \linewidth
    - 8\tabcolsep
    - 5\arrayrulewidth % if package array is loaded
  \relax

\end{document}

